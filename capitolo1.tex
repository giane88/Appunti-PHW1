\label{capitolo1}
\section{Introduzione}
\subsection{Progettazione Hardware 1}
Nel corso di progettazione hardware 1 si tratter� la progettazione di sistemi complessi, non � solo a livello di VHDL, ma in realt� si tratteranno le metodologie di progettazione hardware e gli algoritmi per tale progettazione.\\
Si tratteranno diversi aspetti del flusso di progettazione ovvero:
\begin{itemize}
\item Sintesi dei sistemi digitali: progettazione e costruzione del sistema
\item Collaudo: fase tipica della progettazione hardware che non coincide con la verifica della corretta funzionalit� del sistema rispetto alle specifiche.
\item Verifica dei sistemi: questa � la vera fase di debug e di validazione rispetto alle specifiche.
\end{itemize}
I prerequisiti del corso sono la conoscienza degli argomenti trattati durante il corso di Reti Logiche. 
L'esame si compone di una prova scritta della durata di un'ora e un quarto, composta da 4/5 esercizi per un totale di 32 punti.
\subsection{Progettazione Hardware 2}
Nato da un corso tenuto da docenti di architetture, tenta di presentare lo sviluppo di hardware e software partendo da una specifica che non definisce la distribuzione dei compiti tra hardware e software.\\
Dove per software si parla di sistema dedicato. Mentre per hardware si intende una periferica (acceleratori hardware).
Gli argomenti trattati nel corso sono:
\begin{itemize}
\item Fase di specifica e modellazione di un sistema: modellazione di un sistema fisico di tipo digitale (che cos'� un modello di computazione e comunicazione?) prestanto particolare attenzione ai problemi di determinismo, parallelismo e concorrenza.
\item Progettazione e co-design: progettazione contemporanea di hardware e software (Problematiche di partizionamento, mapping delle risorse da usare).
\item Co-simulazione a livello di sistema (dipende dal tempo a disposizine del corso).
\end{itemize}
Preriquisiti di progettazione hardware 2 sono gli argomenti trattati in architetture avanzate dei calcolatori e reti logiche.
L'esame � composto da un test a risposta multipla che genera met� della valutazione finale mentre la seconda met� sar� assegnata previa realizzazione di un progetto che pu� essere di due tipi:
\begin{itemize}
\item Sviluppo di uno componente da integrare in un software per la progettazione assistita (CAD) da sviluppare in cpp.
\item Ricerca bibliografica su un argomento assegnato dal docente.
\end{itemize} 
\section{Introduzione PHW1}
Negli ultimi anni la complessit� dei progetti hardware � di molto incrementata, mentre i tempi per la loro realizzazione si sono ridotti drasticamente.
Per sopperire a queste mancanze la grandezza dei gruppi di lavoro � aumentata in maniera esponenziale, con progettisti con diverse capacit� che lavorano in parallelo a diversi livelli di astrazione.
La gestione delle complessit� e delle comunicazioni di progetto � arrivata ad un punto critico e gli strumenti di progettazione e di sintesi anche se molto evoluti sono inadeguati allo sviluppo di tali progetti costringendo il progettista ad usare fino a 50 strumenti differenti.
La sfida principale nella progettazione di un nuovo hardware si focalizzano principalmente sull'individuare il giusto compromesso tra time-to-market e costo/prestazioni.
La simulazione � ancora il principale mezzo per la verifica funzionale del progetto ma molto spesso � inadeguata rispetto alle dimensioni dello stesso.\\
Vi � un thrade of per quanto riguarda i modelli ad alto livello e quelli dettagliati; i modelli ad alto livello sono facili da mantenere ma omettono una grande quantit� di informazioni rendendo cos� le simulazioni molto veloci. I modelli pi� dettagliati richiedono un partizionamento maggiore aumentando i costi di comunicazione, inoltre, pur essendo pi� facili da maneggiare e ottimizzare richiedono tempi di simulazione molto lunghi.
Per questo i progettisti sono soliti dividere il problema in sottoproblemi pi� semplici e facili da controllare, dal sottoproblema individuano una soluzione tenedo conto delle condizioni di contorno e costruendo delle interfacce per lo scambio dei dati tra i componenti.\\
Una buona integrazione tra gli strumenti di progettazione e metodologie di progetto comporta comporta indirizzare la complessit� algoritmica mediante la suddivisione da parte dei progettisti del problema e l'indirizzamento degli strumenti automatici in base all'esperienza dei progettisti.\\

